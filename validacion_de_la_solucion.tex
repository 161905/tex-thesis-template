\secnumbersection{VALIDACIÓN DE LA SOLUCIÓN}

Se debe validar la solución propuesta. Esto significa probar o demostrar que la solución propuesta es válida para el entorno donde fue planteada.

Tradicionalmente es una etapa crítica, pues debe comprobarse por algún medio que vuestra propuesta es básicamente válida. En el caso de un desarrollo de software es la construcción y sus pruebas; en el caso de propuestas de modelos, guías o metodologías podrían ser desde la aplicación a un caso real hasta encuestas o entrevistas con especialistas; en el caso de mejoras de procesos u optimizaciones, podría ser comparar la situación actual (previa a la memoria) con la situación final (cuando la memoria está ya implementada) en base a un conjunto cuantitativo de indicadores o criterios.

\subsection{EJEMPLO DE COMO CITAR TABLAS}

Se colocó una tabla que se puede referenciar también desde el texto (Ver tabla \ref{table:coloquios}).

\begin{table}[h]
    \centering
    \caption{\label{table:coloquios} Coloquios del Ciclo de Charlas Informática.} Fuente: Elaboración Propia.
    \begin{tabular}{|p{7cm}|p{7cm}|}
        \hline
        Título Coloquio & Presentador, País \\
        \hline
        ``Sensible, invisible, sometimes tolerant, heterogeneous, decentralized and interoperable... and we still need to assure its quality...''' & Guilherme Horta Travassos, Brasil.\\
        \hline
        ``Dispersed Multiphase Flow Modeling: From Environmental to Industrial Applications''' & Orlando Ayala, EE.UU.\\
        \hline
        ``Líneas de Producto Software Dinámicas para Sistemas atentos el Contexto''' & Rafael Capilla, España.\\
        \hline
        ... & ... \\
        \hline
    \end{tabular}
\end{table}
